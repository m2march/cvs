\documentclass[a4paper,11pt]{article}
\usepackage[T1]{fontenc}
\renewcommand*\familydefault{\sfdefault}
\usepackage[vmargin=3cm, hmargin=3cm]{geometry}
\usepackage[spanish]{babel}
\usepackage[numbers, sort]{natbib}

\setcounter{page}{7}

\title{\vspace{-3cm}Plan de Investigación  \\ Cargo JTP Dedicación Exclusiva \\
{\large Martin A. Miguel - LU 181/09 - N. Legajo 0167777}}
\date{}

\begin{document}
\maketitle

\vspace{-2cm}
\section*{Introducción/Motivación} 

El presente plan de investigación propone el desarrollo de modelos de
inteligencia artifical que, a partir de un estímulo musical simbólico, estimen
para distintos momentos del mismo la certeza que tiene un oyente humano en su
predicción de cómo continuará el mismo.

Las producciones artísticas o de entretenimiento deben mantener un balance
entre la novedad y la familiaridad \cite{berlyne1971aesthetics}. Esto es
particularmente evidente en la música, donde el uso de repeticiones y
estructuras de organización es un recurso común y fundamental. En este ámbito,
un estímulo musical propone una estructura a partir de la cual un oyente puede
generar predicciones sobre como continuará. Por otra parte, la música también
tiene desvíos de la estructura propuesta a fin de generar sorpresa. Sin no
estuviera establecida esta estructura, no habría sensación de sorpresa ya que
el oyente no establecería predicciones en primer lugar.  De esta forma, un
estímulo musical debe balancear la novedad y la familiaridad, de forma que un
oyente pueda generar predicciones que sean luego desafiadas
\cite{huron2010musical, vuust2018incongruity}.

En este contexto, este trabajo propone desarrollar modelos de inteligencia
artificial y neurociencia computacional que permitan estimar el grado de certeza
de un oyente frente a un estímulo musical en distintos puntos del mismo.
Estas herramientas son de utilidad para proveer información a compositores y
enriquecer modelos de composición automática. Este análisis también puede
funcionar como entrada adicional a otras tareas del campo de recuperación de
información musical (\emph{Music Information Retrieval} o \emph{MIR}, en
inglés), como ser segmentación o clasificación automática, así como sistemas de
recomendación.

Desde las áreas de inteligencia artificial, procesamiento de señales y MIR se
han desarrollado numerosos modelos computacionales que buscan comprender un
estímulo musical. Ejemplo de tareas donde esto sucede es composición
\cite{briot2020deep}, clasificación de emociones \cite{downie2008mood} y
segmentación automática \cite{mcfee2017evaluating} de estímulos musicales. No
obstante, estos trabajos se enfocan en resolver la tarea en cuestión y no
en reflejar la forma en que un oyente humano procesa la música. De esta forma,
no son herramientas para estimar la certeza de un oyente. 

IDyOM es un modelo estadístico que fue desarrollado para la tarea de estimación
de certeza en un estímulo musical \cite{pearce2005construction}. El mismo
permite estimar la probabilidad de distintas continuaciones a un segmento de un
estímulo musical. Para ello se basa en regularidades estadísticas aprendidas a
partir de un cuerpo de datos. Dada la distribución de probabilidad de las
continuaciones, la certeza del oyente se estima a partir la entropía de la
misma. La distribución de probabilidad estimada por el modelo, así como la
estimación de certeza, fueron contrastadas con datos obtenidos de oyentes
humanos \cite{hansen2014uncertainty}. En los experimentos, los
participantes debían, para un contexto musical dado, puntuar el nivel de
incertidumbre que tenían para posibles continuaciones. También, luego del
contexto, eran presentados con distintas continuaciones y debían reportar que
tan inesperadas eran las mismas. La estimación de probabilidad de IDyOM para
las continuaciones mostró correlaciones significativas con las medidas de
sorpresa de los participantes (r = 0.695, p < 0.01). De igual manera, la
estimación de entropía del modelo correlacionó significativamente con los
reportes de incertidumbre de los participantes (r = 0.466, p = 0.02).

El funcionamiento de IDyOM se basa en cadenas de Markov de orden variable,
donde la distribución de probabilidad de una continuación para un contexto
musical se establece a partir de contabilizar cuantos contextos similares al
evaluado están presentes en los datos de entrenamiento y poseen la continuación
en cuestión. Este mecanismo de inferencia puede ser extendido considerando
otros mecanismos de aprendizaje y resumen que son utilizados en la cognición
humana. Se considera que la misma es jerárquica, manteniendo concepciones del
mundo de distintos niveles de abstracción \cite{friston2010free}. Por ejemplo,
en el procesamiento de secuencias, las personas mantienen distribuciones de
transición (similares a las cadenas de Markov), pero también realizan procesos
de agrupación de la secuencia en secciones relevantes. Asimismo, pueden
abstraer patrones algebraicos abstractos que luego se instancian con valores
específicos (por ejemplo, la secuencia abstracta $AAB$, que puede instanciarse
en $112$ o $225$) y abstraer estas aún más en árboles de parseo
\cite{dehaene2015neural}.

\section*{Objetivos} 

Este plan de investigación propone desarrollar modelos para la estimación de la
certeza a lo largo del tiempo que tendría un oyente humano sobre las
continuaciones de un estímulo musical dado un contexto.

En primer lugar, se propone la extensión de las ideas propuestas en el
modelo IDyOM \cite{pearce2005construction} con nuevos mecanismos de inferencia
jerárquica como los mencionados en la sección antecedentes
\cite{dehaene2015neural}. 

En segundo lugar, se propone reutilizar modelos de composición automática
basados en redes neuronales que aprenden recurrencias estadísticas en estímulos
musicales para obtener estimadores de certeza. Esto puede realizarse a partir
de métricas obtenidas de las activaciones de la red neuronal, así como
reutilizando las capas inferiores de las mismas y agregando una nueva capa
superior entrenada para esta nueva tarea (proceso conocido como
\emph{fine-tunning}).

En tercer lugar, se propone realizar un experimento para recolectar información
de certeza en contextos musicales. Estos datos serán utilizados para evaluar
los modelos así como para mejorar el proceso de entrenamiento al contar con
mayor cantidad de datos.

\section*{Metodología de trabajo}

Para extender las propuestas de IDyOM, se propone el uso de modelos de
inferencia Bayesiana jerárquica que utilizan gramáticas para describir los
datos de entrada \cite{tenenbaum2011grow}. Por un lado, el uso de inferencia
bayesiana tiene ineherentemente una estimación de probabilidad y por lo tanto
de certeza.  Por el otro, el uso de gramáticas permite hacer descripciones de
secuencias de datos basadas en árboles. Estas gramáticas, por ejemplo, permiten
reflejar la forma jerárquica en la que se anota la música escrita en
partituras, donde los elementos constituyentes (las notas) se agrupan de forma
recursiva \cite{fitch2013meter-vs-pulse}.  Esta técnica de modelado ha mostrado
ser flexible para representar distintos tipos de datos (taxonomías de animales,
ubicaciones espaciales, distancia entre colores) y obtener información
relevante a partir de pocos datos \cite{tenenbaum2011grow}. Asimismo, el
modelado del aprendizaje utilizando gramáticas se ha utilizado para reflejar
características de procesamiento humano, como ser tiempo de respuesta o
dificultad de la tarea \cite{tano2018learning}.

Para obtener métricas a partir de modelos de redes neuronales, proponemos
inspeccionar modelos de composición automática secuenciales basados en redes
neuronales. Estos modelos
son entrenados para, dado un contexto musical, estimar una distribución de
probabilidad de continuaciones de forma que repliquen el dataset
\cite{huang2018music}. De esta forma, y al igual que en IDyOM, la certeza del
oyente puede estimarse a partir de la entropía de esta distribución de
probabilidad. Esta metodología fue utilizada con éxito en un trabajo previo
para obtener un estimador de la certeza del pulso musical
\cite{pironio2021clarity}.

Para el experimento, es posible realizar tareas donde los participantes
reportan su certeza respecto de la continuación de distintos contextos
musicales \cite{hansen2014uncertainty}. Para tomar medidas durante el proceso
de escucha sin inducir una pausa, es posible utilizar mediciones fisiológicas
como pupilometría \cite{zhang2020using}, 

\section*{Descripción del grupo de investigación}

El trabajo de investigación se realizará dentro del Laboratorio de Inteligencia
Artificial Aplicada. El mismo se enfoca en la aplicación de técnicas de
inteligencia artificial y aprendizaje automático a distintos problemas. En
particular, se mantienen líneas de investigación donde se desarrollan modelos
que buscan reflejar y predecir características de la cognición humana (como ser
la trayectoria de movimientos oculares en la percepción de imágenes
\cite{bujia2022modeling}, estimación de riesgo de psicosis
\cite{bedi2015psychosis} o predictibilidad de palabras en un texto
\cite{bianchi2020predictability}).

\section*{Factibilidad}

Para el desarrollo de los modelos de inferencia no es necesario equipo
especializado. Para el trabajo con redes neuronales puede ser necesario, según
si se deben entrenar modelos, computadoras equipadas con placas de vídeo, las
cuales están disponibles en el laboratorio. Para la realización de experimento
se requiere una sala donde conducirlos. El laboratorio cuenta con una sala
designada para ello. Además cuenta con una cámara de seguimiento ocular en caso
de recolectar datos de pupilometría.

\section*{Otros}

Se espera publicar el trabajo de modelado utilizando modelado basado en
gramáticas en la revista de investigación técnica en música \emph{Journal of
New Music Research}. Se espera publicar el trabajo realizado a partir de
redes neuronales en la conferencia de la \emph{International Society of Music
  Information Retrieval} (ISMIR). Los datos experimentales recolectados podrán
  ser publicados en la misma conferencia.

Algunas secciones de este plan pueden ser llevadas a cabo en conjunto con un
estudiante, ya sea como tesis de grado o beca de investigación. Un ejemplo es
la exploración sobre modelos de redes neuronales, donde, dado un modelo ya
seleccionado y un conjunto de datos de evaluación establecido, la/el estudiante
deberá familiarizarse con el modelo, agregar código para obtener métricas del
mismo, realizar una evaluación de las mismas y presentar los resultados.

\noindent\begin{tabular}{p{0.45\textwidth} p{0.45\textwidth}}
    \vspace{1.5cm} & \\
    % & \includegraphics[height=40pt]{firma_mmiguel.png} 
    %  & \includegraphics[height=40pt]{firma_dfslezak_small.jpg} \\
       \rule[0pt]{2in}{0.5pt} & 
        \rule[0pt]{2in}{0.5pt}\\
        Postulante & Profesor \\
        (Martin A. Miguel) &  (Diego Fernandez Slezak)
        \end{tabular}


{\small
\bibliography{ref-index}
\bibliographystyle{unsrtnat}
}
\end{document}
