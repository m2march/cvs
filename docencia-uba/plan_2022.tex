\documentclass[a4paper,11pt]{article}
\usepackage[T1]{fontenc}
\renewcommand*\familydefault{\sfdefault}
\usepackage{geometry}
\usepackage{natbib}

\title{Plan de Investigación  \\ Cargo JTP Dedicación Exclusiva \\
{\large Martin A. Miguel - LU 181/09 - N. Legajo 167777}}
\date{}

\begin{document}
\maketitle

\section*{Introducción/Motivación} 

El presente plan de investigación propone el desarrollo de modelos de
inteligencia artifical que, a partir de un estímulo musical simbólico, estimen
para distintos momentos del mismo la certeza que tiene un oyente humano en su
predicción de cómo continuará el mismo.

Las producciones artísticas o de entrenimiento como son las obras artísticas, y
en particular la música, deben mantener un balance entre la novedad y la
familiaridad \cite{berlyne1971aesthetics}. Esto es particularmente evidente en
la música, donde el uso de repeticiones y estructuras de organización
es un recurso común y fundamental. En este ámbito, un estímulo musical propone
una estructura a partir de la cual un oyente puede generar predicciones sobre
cómo continuará. Por otra parte, este mismo también tiene desvíos de la
estructura propuesta a fin de generar sorpresa. Sin no estuviera establecida
esta estructura, no habría sensación de sorpresa ya que el oyente no
establecería predicciones en primer lugar.  De esta forma, un estímulo musical
debe balancear la novedad y la familiaridad, de forma que un oyente pueda
generar predicciones que sean luego desafiadas \cite{huron2010musical,
vuust2018incongruity}.

En este contexto, este trabajo propone desarrollar modelos de inteligencia
artifical y neurociencia computacional que permitan estimar el grado de certeza
de un oyente frente a un estímulo musical en distintos puntos del mismo.
Estas herramientas son de utilidad para proveer información a compositores y
enriquecer modelos de composición automática. Este análisis también puede
funcionar como entrada adicional a otras tareas del campo de recuperación de
información musical (\emph{Music Information Retrieval} o \emph{MIR}, en
inglés), como ser segmentación o clasificación automática o sistemas de
recomendación.

Desde las áreas de inteligencia artifical, procesamiento de señales y MIR se
han desarrollado numerosos modelos computacionales que buscan comprender un
estímulo musical. Ejemplo de tareas donde esto sucede es composición
\cite{briot2020deep}, clasificación de emociones \cite{downie2008mood} y
segmentación automática \cite{mcfee2017evaluating} de estímulos musicales. No
obstante, estos trabajos se enfocan en resolver la tarea en cuestión y no
en reflejar la forma en que un oyente humano procesa la música. De esta forma,
no son herramientas para estimar la certeza de un oyente. 

IDyOM es un modelo estadístico que fue desarrollado para la tarea de estimación
de certeza en un estímulo musical \cite{pearce2005construction}. El mismo
permite estimar la probabilidad de distintas continuaciones a un segmento de un
estímulo musical. Para ello se basa en regularidades estadísticas aprendidas a
partir de un cuerpo de datos. Dada la distribución de probabilidad de las
continuaciones, la certeza se estima a partir la entropía de la misma.


\section*{Objetivos} 
    (a encarar suponiendo un horizonte de investigación de tres años)

\section*{Metodología de trabajo}
describa brevemente cómo llevará adelante la
investigación. Si corresponde, describa las herramientas o equipamiento a utilizar.

\section*{Descripción del grupo de investigación}
en el que se inserta o se insertará.

\section*{Factibilidad}

Describir equipamiento a utilizar y, de existir y/o
corresponder, los subsidios
de investigación que sostengan este desarrollo.

\section*{Otros}

Cualquier otro elemento que considere relevante en el contexto del
plan. Por ejemplo, su plan de publicación de resultados (conferencias o revistas en las
cuales espera poder enviar sus resultados).

\section*{Referencias}
\bibliography{references-index}
\bibliographystyle{plain}
\end{document}
