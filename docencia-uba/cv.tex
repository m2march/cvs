\documentclass[a4paper,10pt]{article}

\usepackage{mathptmx}
\usepackage[centertags]{amsmath}
\usepackage{amsfonts}
\usepackage{amssymb}
\usepackage{amsthm}
\usepackage[colorlinks,bookmarks=true]{hyperref}
\usepackage[utf8]{inputenc}
\usepackage{url}
\usepackage{fullpage}
\usepackage{eurosym}
\usepackage[spanish]{babel}
\usepackage{fixfoot}
\usepackage{todonotes}
\usepackage{multicol}

\hypersetup{
  pdftitle = {cv},
  pdfkeywords = {resume, cv, computer science},
  pdfauthor = {}
}

\begin{document}

\begin{center}
    { \Huge
    Curriculum Vitae}
\bigskip

{\Large Martin Alejandro Miguel} \\
\medskip

\emph{mmiguel@dc.uba.ar} \\
DNI: 35.254.343 | LU: 181/09 | Legajo: 0167777 | CUIT: 20-35254343-9
\end{center}


\section{Antecedentes Docentes}

\subsection{Universitarios}

\begin{itemize}
    \item Jefe de Trabajos Prácticos (dedicación simple), Universidad de Buenos
        Aires, Facultad de Ciencias Exactas y Naturales, Departamenteo de
        Computación. \emph{2do cuatrimestre 2017 hasta 1er cuatrimestre 2022 -
        Algoritmos y Estructuras de Datos 2.}
    \item Ayudante de 1ra (dedicación simple), Universidad de Buenos Aires,
        Facultad de Ciencias Exactas y Naturales, Departamenteo de Computación.
        \emph{Año 2016 y 1er cuatrimestre 2017 - Algoritmos y Estructuras de
        Datos 2.}
    \item Ayudante de 2da, Universidad de Buenos Aires, Facultad de Ciencias
        Exactas y Naturales, Departemento de Computación. \emph{1er
        cuatrimestre 2012 - Algoritmos y Estructuras de Datos 1.}
    \item Ayudante de 2da, Universidad de Buenos Aires, Facultad de Ciencias
        Exactas y Naturales, Departemento de Computación. \emph{1er y 2do
        cuatrimestre 2011 - Algoritmos y Estructuras de Datos 2.}
\end{itemize}

\section{Antecedentes Científicos}

\subsection{Publicaciones con Arbitraje}

\subsubsection{En revistas}
\begin{itemize}
  \item \emph{Kiss, Luca; Guiot, Cecilia; Hashim, Sarah; D’Aleman Arango,
    Nicolas; Miguel, Martin A.}
    \textbf{The 14th
    International Conference of Students of Systematic Musicology (SysMus21)}
    (2022). Music \& Science. doi:10.1177/20592043221076613.

  \item\emph{Miguel, Martin A.; Riera, Pablo; Fernandez Slezak, Diego.} 
    \textbf{A simple and cheap setup for timing tapping responses synchronized
    to auditory stimuli.} (2021) Behav Res 54, p. 712–728. https://doi.org/10.3758/s13428-021-01653-y

  \item 
    \emph{Miguel, Martín A.; Sigman, Mariano; Fernandez Slezak, Diego.}
    \textbf{From beat tracking to beat expectation: Cognitive-based beat
    tracking for capturing pulse clarity through time} 
    \emph{(2020) PLOS ONE 15(11): e0242207.
        https://doi.org/10.1371/journal.pone.0242207, p.1-22}
    \item 
\emph{Belloli, Laouen; Miguel, Martín A.; Goldin, Andrea P., Fernandez Slezak,
        Diego}.
\textbf{Mate Marote: a BigData platform for massive scale educational
        interventions}.
        \emph{AGRANDA 2016-JAIIO 45, 2016, Tres De Febrero, Buenos Aires,
        Argentina}, (ISSN: 2451-7569), p. 107-114.
    \item 
\emph{Martin A. Miguel}.
\textbf{Inferencia de Tactus con Fundamentos Estadísticos para Tap-dancing}.
        \emph{ASAI 2015-JAIIO 44, 2015, Rosario, Argentina}, (ISSN: 2451-7585),
        p. 168-175.
\end{itemize}

\subsubsection{En conferencias}
\begin{itemize}
 \item\emph{Miguel, Martin A.; Fernandez Slezak, Diego.} 
   \textbf{Modeling beat uncertainty as a 2D distribution of period and
   phase: a MIR task proposal.} (2021) 
   Proc. of the 22nd Int. Society for Music Information Retrieval Conf., Online.

 \item \emph{Pironio, Nicolas; Fernandez Slezak, Diego; Miguel, Martin A.} 
     \textbf{Pulse clarity metrics developed from a deep learning beat tracking
     model.} (2021) Proc. of the 22nd Int. Society for Music Information
Retrieval Conf., Online, 2021.
\end{itemize}

\subsection{Participación en congresos o acontecimientos nacionales o
internacionales}

\begin{itemize}
  \item Presentación de poster: 
      \textbf{Modeling beat ambiguity in period and phase.}
    \emph{Miguel, Martin .A; Fernandez Slezak, Diego.} 
International Conference of Students of Systematic
Musicology 21 (SysMus 21), 2021, Online and Aahrus, Denmark (DOI
  10.17605/OSF.IO/5WRS3)
  
\item Presentación de poster: 
  \textbf{A continuous model of pulse clarity: towards inspecting affect through
  expectations in time.} 
  \emph{Miguel, Martin .A; Sigman, Mariano;  Fernandez Slezak, Diego.} 
  Neuromusic VII, 2021, 
  Online and Aahrus, Denmark. (DOI 10.17605/OSF.IO/FGVB2)

  \item Presentación de poster: 
    \textbf{Evaluation of pulse clarity models on multiple datasets.} 
    \emph{Pironio, Nicolas; Fernandez Slezak, Diego; Miguel, Martin A.} 
Rhythm Perception and Production Workshop 2021 (RPPW 21), 2021, Online and
Oslo, Norway (DOI 10.17605/OSF.IO/SDQ5P)

  \item Presentación de poster: 
  \textbf{Development and evaluation of pulse clarity metrics based on a deep learning
beat tracking model.} 
    \emph{Pironio, Nicolas; Fernandez Slezak, Diego; Miguel, Martin A.} 
  16th International Conference on Music Perception and Cognition (ICMPC 21),
  2021, Online. 

\item Participación virtual a Cross-European Winter School On Musical Ability.
  Febrero 2021, virtual.

\item Participación virtual a 16th Annual McMaster
  NeuroMusic Virtual Conference (NeuroMusic 2020). Noviembre 2020,
  virtual.

\item Participación virtual a 13th International Conference of Students of
  Systematic Musicology (SysMus 20). Septiembre
  2020, Online.

    \item Participación en charla sobre trabajo:
\textbf{Analysis of the behaviour of a beat tracking model to estimate human
        perception of task difficulty}. Pironio, Nicolas; Miguel Martín A. 49-JAIIO, 2020, Argentina.
    \item Asistencia a la reunión Latino-Americana de Inteligencia Artifical,
        KHIPU 2019, Montevideo, Uruguay.
    \item Presentación de poster:
        \textbf{A continuous model of pulse clarity}.
        SMPC 2019, New York University, New York, Estados Unidos.
    \item Presentación de poster:
        \textbf{Tapping to your own beat}.
        SMPC 2019, New York University, New York, Estados Unidos.
    \item Presentación de poster:
        \textbf{Beat tracking model for cognitive evaluation}.
        MLSS 2018, Universidad Torcuato Di Tella, Buenos Aires, Argentina.
    \item Presentación de poster: \textbf{Inferencia de Tactus con Fundamentos Estadísticos para
Tap-dancing}.
    ECI 2017, Facultad de Ciencias Exactas y Naturales, Universidad de Buenos
        Aires, Argentina.
    \item Presentación de poster y charla sobre trabajo:
\textbf{Inferencia de Tactus con Fundamentos Estadísticos para
        Tap-dancing}. 44-JAIIO, 2015, Rosario, Argentina.
    \item Participación como voluntario en IJCAI-15, Buenos Aires, Argentina.
\end{itemize}

\subsection{Formación de Recursos Humanos}

\begin{itemize}
\item Mentoría del estudiante de grado Lucas Somacal en el contexto de 
del programa de Becas de Iniciación a la investigación en Ciencias de la
Computación (BIICC), otorgadas por el Departamento de Computación, Facultad
de Ciencias Exactas y Naturales, Universidad de Buenos Aires, Buenos Aires,
  Argentina. La pasantía se realizó durante el año 2021
y fue titulada: Exploración
de transferencia de estilo en música simbólica usando espacios latentes en
  VAEs.

\item
Mentoría del estudiante de grado Nicolás Pironio en el contexto de 
del programa de Becas de Iniciación a la investigación en Ciencias de la
Computación (BIICC), otorgadas por el Departamento de Computación, Facultad
de Ciencias Exactas y Naturales, Universidad de Buenos Aires, Buenos Aires,
  Argentina. La pasantía se realizó durante el año 2020
y fue titulada: Analisis de la reutilización de un modelo de seguimiento de
pulso basado en redes neuronales para estimación de claridad del pulso.
\end{itemize}

\section{Antecedentes de Extensión}

\begin{itemize}
    \item 
      Participación como expositor de una charla ilustrando el perfil de los
      egresados y egresadas de la carrera de Ciencias
      de la Computación. “Semana de la Computación 2019”.
      Facultad de Ciencias Exactas y Naturales, UBA.
    \item Participación como docente del taller de programación musical en la
        “Semana de la Computación 2018”.
        Facultad de Ciencias Exactas y Naturales, UBA.
    \item Participación en la organización del taller de programación musical
        en la “Semana de la Computación 2017”. Facultad de Ciencias Exactas y
        Naturales, UBA. 
    \item Participación como coordinador de charlas en la “Semana de la
        Computación 2016”. Facultad de Ciencias Exactas y Naturales, UBA. 
\end{itemize}

\section{Antecedentes Profesionales}

\subsection{Actividades profesionales fuera del ámbito académico}

\begin{itemize}
    \item {Data Scientist, Avenida.com. Buenos Aires, Argentina.
            Enero 2016 - Marzo 2016. \\
            {\small \itshape Mejoras en el sistema de búsqueda del sitio.}}

    \item {Programador, Proyecto MateMarote, UBA. Buenos Aires, Argentina.
            Abril 2015 - Diciembre 2015. \\
            {\small \itshape Desarrollo de sistema online para soporte de
        juegos en internet (Java) y desarrollo de juegos de entrenamiento
cognitivo para niños (javascript)}}

    \item {Ingeniero de Software (Pasantía), Google.com. New York, USA. Enero
            2014 - Abril 2014. \\ 
    {\small \itshape Desarrollo y extensión de frameworks de testing para tests
de performance, tests funcionales y tests de regresión.}}

    \item {Programador Java, Despegar.com. Buenos Aires, Argentina. Agosto 2012
           - Diciembre 2013. \\
    {\small \itshape Desarrollo de componentes para integrar en un sistema de
software de mayor escala. Desarrollo de aplicaciones web y frameworks
utilitarios.}}

    \item {Programador Java Jr. (J2ME / Blackberry), SenseByte. Buenos Aires,
            Argentina. Enero 2009 - Enero 2010. \\ 
    {\small \itshape Desarrollador de apliaciones stand-alone y
cliente-servidor.  Desarrollo de interfaces con hardware no estandar utilizado
por las aplicaciones desarrolladas.}}

\end{itemize}

\section{CALIFICACIONES, TITULOS, ESTUDIOS, OTROS}

\subsection{Títulos Obtenidos}

\begin{itemize}
    \item {Lic. en Ciencias de la Computación, Facultad de Ciencias Exactas y
        Naturales, Buenos Aires, Argentina. 2008 - 2015.}
    \item {Título Secundario, Instituto SUMMA, Buenos Aires, Argentina. 2003
            - 2007. \\
        {\small \itshape Bachiller con Orientación a Comunicaciones.}}
\end{itemize}

\subsection{Carrera de doctorado}

\textbf{Doctorado en Ciencias de la Computación}\\
\indent{}\emph{Facultad de Ciencias Exactas y Naturales, Universidad de Buenos Aires.}
\begin{itemize}
    \item \emph{Director de Tesis}: Diego Fernandez Slezak
    \item \emph{Co-director de Tesis}: Mariano Sigman
    \item \emph{Tema de Tesis}: Inferencia de estructuras y patrones temporales
        basados en la cognición musical
    \item \emph{Fecha de ingreso}: 5/12/2016
    \item \emph{Estado de avance}: Manuscrito final entregado en Mayo de 2022,
      actualmente bajo revisión de los jurados.
\end{itemize}

\subsection{Becas}

\begin{itemize}
    \item Beca Doctorado CONICET, con inicio el 01/04/2016 y fin el 01/04/2022
      (año extra a razón de la pandemia de COVID-19).
\end{itemize}

\subsection{Otros elementos}

\subsubsection{Calificaciones}

Para la Licenciatura en Ciencias de Computación del Departamento de
Computación, FCEyN, UBA.

\begin{tabular}{ l c }

\textbf{Obligatorias} \smallskip  & \\ 
Análisis Numérico & 9\\
Algebra & 5\\
Probabilidad y Estadística & 10\\
Algoritmos y Estructuras de Datos I & 10\\
Algoritmos y Estructuras de Datos II & 10\\
Algoritmos y Estructuras de Datos III & 9\\
Organización del Computador I & 8\\
Organización del Computador II & 8\\
Sistemas Operativos & 10\\
Métodos Numéricos & 10\\
Ingeniería de Software I & 7\\
Ingeniería de Software II & 9\\
Teoría de las Comunicaciones & 10\\
Bases de Datos & 9\\
Lógica y Computabilidad & 9\\
Teoría de Lenguajes & 10\\
Paradigmas de Lenguajes de Programación & 10\\
\end{tabular}

\medskip
\begin{tabular}{l c}
\textbf{Optativas} \smallskip & \\
Redes Neuronales & 9\\
Introducción a Tecnologías del Habla & 9\\
Teoría de Juegos & Solo cursada\\
Desarrollo de Sistemas Operativos & 10\\
Aprendizaje Automático & 10\\
\end{tabular}
\bigskip

Escala de notas: 10

Promedio: 9.09

\bigskip

Para la carrera de Doctor en Ciencias de la Computación:

\medskip
\begin{tabular}{l c}
Ciencia de Datos R: Fundamentos Estadísticos & 10 \\
Fundamentos de Inferencia Bayesiana & 10 \\
Introducción a la Ciencia de los Datos & 10 \\ 
Procesamiento de Señales & 10 \\
Integración de Bases de Conocimiento & 10 \\
Introducción a la Neurociencia Cognitiva Computacional & 10 \\
\end{tabular}


\end{document}
