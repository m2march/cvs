\documentclass[a4paper,10pt]{article}

\usepackage{mathptmx}
\usepackage[centertags]{amsmath}
\usepackage{amsfonts}
\usepackage{amssymb}
\usepackage{amsthm}
\usepackage[colorlinks,bookmarks=true]{hyperref}
\usepackage[utf8]{inputenc}
\usepackage{url}
\usepackage{fullpage}
\usepackage{eurosym}
\usepackage[spanish]{babel}
\usepackage{fixfoot}

\DeclareFixedFootnote{\enpublicados}{Enunciada en sección de publicaciones}

\hypersetup{
  pdftitle = {cv},
  pdfkeywords = {resume, cv, computer science},
  pdfauthor = {}
}

\begin{document}

\begin{center}
    { \Huge
    Curriculum Vitae}
\bigskip

{\Large Martin Alejandro Miguel} \\
\medskip

\emph{m2.march@gmail.com} \\
DNI: 35.254.343 | LU: 181/09 
\end{center}


\section{Antecedentes Docentes}

\subsection{Universitarios}

\begin{itemize}
    \item Ayudante de 2da, Universidad de Buenos Aires, Facultad de Ciencias
Exactas y Naturales, Departemento de Computación. \emph{1er y 2do cuatrimestre
2011 - 
Algoritmos y Estructuras de Datos 2.}
    \item Ayudante de 2da, Universidad de Buenos Aires, Facultad de Ciencias
Exactas y Naturales, Departemento de Computación. \emph{1er cuatrimestre 2012 -
Algoritmos y Estructuras de Datos 1.}
\end{itemize}

\section{Antecedentes Científicos}

\subsection{Publicaciones con Arbitraje}

\begin{itemize}
    \item 
\emph{Martin A. Miguel}, 
\textbf{Inferencia de Tactus con Fundamentos Estadísticos para Tap-dancing}.
44-JAIIO, 2015, Rosario, Argentina (\emph{En prensa}).
\end{itemize}

\subsection{Participación en congresos o acontecimientos nacionales o
internacionales}

\begin{itemize}
    \item Presentación de poster y charla para trabajo 
\textbf{Inferencia de Tactus con Fundamentos Estadísticos para
Tap-dancing}.\enpublicados{}
44-JAIIO, 2015, Rosario, Argentina (\emph{En prensa}).
    \item Participación como voluntario en IJCAI-15, Buenos Aires, Argentina.
\end{itemize}

\section{Antecedentes de Extensión}

\section{Antecedentes Profesionales}

\subsection{Actividades profesionales fuera del ámbito académico}

\begin{itemize}
    \item {Programador, Proyecto MateMarote, UBA, Buenos Aires, Argentina.
            Abril 2015 - Actualidad. \\
            {\small \itshape Desarrollo de sistema online para soporte de
        juegos en internet (Java) y desarrollo de juegos de entrenamiento
cognitivo para niños (javascript)}}

    \item {Ingeniero de Software (Pasantía), Google.com. New York, USA. Enero
            2014 - Abril 2014. \\ 
    {\small \itshape Desarrollo y extensión de frameworks de testing para tests
de performance, tests funcionales y tests de regresión.}}

    \item {Programador Java, Despegar.com. Buenos Aires, Argentina. Agosto 2012
           - Diciembre 2013. \\
    {\small \itshape Desarrollo de componentes para integrar en un sistema de
software de mayor escala. Desarrollo de aplicaciones web y frameworks
utilitarios.}}

    \item {Programador Java Jr. (J2ME / Blackberry), SenseByte. Buenos Aires,
            Argentina. Enero 2009 - Enero 2010. \\ 
    {\small \itshape Desarrollador de apliaciones stand-alone y
cliente-servidor.  Desarrollo de interfaces con hardware no estandar utilizado
por las aplicaciones desarrolladas.}}

\end{itemize}

\section{CALIFICACIONES, TITULOS, ESTUDIOS, OTROS}

\subsection{Títulos Obtenidos}

\begin{itemize}
    \item {Título Secundario, Instituto SUMMA, Buenos Aires, Argentina. 2003
            - 2007. \\
        {\small \itshape Bachiller con Orientación a Comunicaciones.}}
\end{itemize}

\subsection{Otros elementos}

\subsubsection{Calificaciones}

Para la Licenciatura en Ciencias de Computación del Departamento de
Computación, FCEyN, UBA.

\begin{tabular}{ l c }
&\\

\textbf{Obligatorias} \smallskip  & \\ 
Análisis Numérico & 9\\
Algebra & 5\\
Probabilidad y Estadística & 10\\
Algoritmos y Estructuras de Datos I & 10\\
Algoritmos y Estructuras de Datos II & 10\\
Algoritmos y Estructuras de Datos III & 9\\
Organización del Computador I & 8\\
Organización del Computador II & 8\\
Sistemas Operativos & 10\\
Métodos Numéricos & 10\\
Ingeniería de Software I & 7\\
Ingeniería de Software II & 9\\
Teoría de las Comunicaciones & 10\\
Bases de Datos & 9\\
Lógica y Computabilidad & 9\\
Teoría de Lenguajes & 10\\
Paradigmas de Lenguajes de Programación & 10\\
&\\

\textbf{Optativas} \smallskip & \\
Redes Neuronales & 9\\
Introducción a Tecnologías del Habla & 9\\
Teoría de Juegos & Examen final pendiente\\
Desarrollo de Sistemas Operativos & 10\\
Aprendizaje Automático & Examen final pendiente\\
\end{tabular}

\bigskip

Escala de notas: 10

Promedio: 9.05

\end{document}
