\documentclass[a4paper,10pt]{article}

\usepackage{mathptmx}
\usepackage[centertags]{amsmath}
\usepackage{amsfonts}
\usepackage{amssymb}
\usepackage{amsthm}
\usepackage[colorlinks,bookmarks=true]{hyperref}
\usepackage[utf8]{inputenc}
\usepackage{url}
\usepackage{fullpage}
\usepackage{eurosym}
\usepackage[spanish]{babel}
\usepackage{fixfoot}
\usepackage{todonotes}
\usepackage{multicol}

\hypersetup{
  pdftitle = {cv},
  pdfkeywords = {resume, cv, computer science},
  pdfauthor = {}
}

\begin{document}

\begin{center}
    { \Huge
    Curriculum Vitae}
\bigskip

{\Large Martin Alejandro Miguel} \\
\medskip

\emph{mmiguel@dc.uba.ar} \\
DNI: 35.254.343 | LU: 181/09 | Legajo: 0167777
\end{center}


\section{Antecedentes Docentes}

\subsection{Universitarios}

\begin{itemize}
    \item Jefe de Trabajos Prácticos (dedicación simple), Universidad de Buenos
        Aires, Facultad de Ciencias Exactas y Naturales, Departamento de
        Computación. \emph{2do cuatrimestre 2017 y 2018 - Algoritmos y
        Estructuras de Datos 2.}
    \item Ayudante de 1ra (dedicación simple), Universidad de Buenos Aires,
        Facultad de Ciencias Exactas y Naturales, Departamenteo de Computación.
        \emph{Año 2016 y 1er cuatrimestre 2017 - Algoritmos y Estructuras de
        Datos 2.}
    \item Ayudante de 2da, Universidad de Buenos Aires, Facultad de Ciencias
        Exactas y Naturales, Departemento de Computación. \emph{1er y 2do
        cuatrimestre 2011 - Algoritmos y Estructuras de Datos 2.}
    \item Ayudante de 2da, Universidad de Buenos Aires, Facultad de Ciencias
        Exactas y Naturales, Departemento de Computación. \emph{1er
        cuatrimestre 2012 - Algoritmos y Estructuras de Datos 1.}
\end{itemize}

\section{Antecedentes Científicos}

\subsection{Publicaciones con Arbitraje}

\begin{itemize}
    \item 
\emph{Martin A. Miguel}, 
\textbf{Mate Marote: a BigData platform for massive scale educational
        interventions}.
        45-JAIIO, 2016, Buenos Aires, Argentina (ISSN: 2451-7569).
    \item 
\emph{Martin A. Miguel}, 
\textbf{Inferencia de Tactus con Fundamentos Estadísticos para Tap-dancing}.
        44-JAIIO, 2015, Rosario, Argentina (ISSN: 2451-7585).
\end{itemize}

\subsection{Participación en congresos o acontecimientos nacionales o
internacionales}

\begin{itemize}
    \item Presentación de poster 
\textbf{Inferencia de Tactus con Fundamentos Estadísticos para
        Tap-dancing}.\enpublicados{} ECI, 2017, Facultad de Ciencias Exactas y
        Naturales, Universidad de Buenos Aires, Argentina.
    \item Presentación de poster y charla para trabajo 
\textbf{Inferencia de Tactus con Fundamentos Estadísticos para
        Tap-dancing}.\enpublicados{} 44-JAIIO, 2015, Rosario, Argentina.
    \item Participación como voluntario en IJCAI-15, Buenos Aires, Argentina.
\end{itemize}

\section{Antecedentes de Extensión}

\begin{itemize}
    \item Participación en la organización del taller de programación musical
        en la “Semana de la Computación 2017”. Facultad de Ciencias Exactas y
        Naturales, UBA. 
    \item Participación como coordinador de charlas en la “Semana de la
        Computación 2016”. Facultad de Ciencias Exactas y Naturales, UBA. 
\end{itemize}

\section{Antecedentes Profesionales}

\subsection{Actividades profesionales fuera del ámbito académico}

\begin{itemize}
    \item {Data Scientist, Avenida.com, Buenos Aires, Argentina.
            Enero 2016 - Marzo 2016. \\
            {\small \itshape Mejoras en el sistema de búsqueda del sitio.}}

    \item {Programador, Proyecto MateMarote, UBA, Buenos Aires, Argentina.
            Abril 2015 - Actualidad. \\
            {\small \itshape Desarrollo de sistema online para soporte de
        juegos en internet (Java) y desarrollo de juegos de entrenamiento
cognitivo para niños (javascript)}}

    \item {Ingeniero de Software (Pasantía), Google.com. New York, USA. Enero
            2014 - Abril 2014. \\ 
    {\small \itshape Desarrollo y extensión de frameworks de testing para tests
de performance, tests funcionales y tests de regresión.}}

    \item {Programador Java, Despegar.com. Buenos Aires, Argentina. Agosto 2012
           - Diciembre 2013. \\
    {\small \itshape Desarrollo de componentes para integrar en un sistema de
software de mayor escala. Desarrollo de aplicaciones web y frameworks
utilitarios.}}

    \item {Programador Java Jr. (J2ME / Blackberry), SenseByte. Buenos Aires,
            Argentina. Enero 2009 - Enero 2010. \\ 
    {\small \itshape Desarrollador de apliaciones stand-alone y
cliente-servidor.  Desarrollo de interfaces con hardware no estandar utilizado
por las aplicaciones desarrolladas.}}

\end{itemize}

\section{CALIFICACIONES, TITULOS, ESTUDIOS, OTROS}

\subsection{Títulos Obtenidos}

\begin{itemize}
    \item {Lic. en Ciencias de la Computación, Facultad de Ciencias Exactas y
        Naturales, Buenos Aires, Argentina. 2008 - 2015.}
    \item {Título Secundario, Instituto SUMMA, Buenos Aires, Argentina. 2003
            - 2007. \\
        {\small \itshape Bachiller con Orientación a Comunicaciones.}}
\end{itemize}

\subsection{Carrera de doctorado}

\textbf{Doctorado en Ciencias de la Computación}\\
\indent{}\emph{Facultad de Ciencias Exactas y Naturales, Universidad de Buenos Aires.}
\begin{itemize}
    \item \emph{Director de Tesis}: Diego Fernandez Slezak
    \item \emph{Co-director de Tesis}: Marinano Sigman
    \item \emph{Tema de Tesis}: Inferencia de estructuras y patrones temporales
        basados en la cognición musical
    \item \emph{Fecha de Ingreso}: 5/12/2016
\end{itemize}

\subsection{Becas}

\begin{itemize}
    \item Beca Doctorado CONICET, con inicio el 01/04/2016 y fin el 01/04/2021
\end{itemize}

\subsection{Otros elementos}

\subsubsection{Calificaciones}

Para la Licenciatura en Ciencias de Computación del Departamento de
Computación, FCEyN, UBA.

\begin{tabular}{ l c }

\textbf{Obligatorias} \smallskip  & \\ 
Análisis Numérico & 9\\
Algebra & 5\\
Probabilidad y Estadística & 10\\
Algoritmos y Estructuras de Datos I & 10\\
Algoritmos y Estructuras de Datos II & 10\\
Algoritmos y Estructuras de Datos III & 9\\
Organización del Computador I & 8\\
Organización del Computador II & 8\\
Sistemas Operativos & 10\\
Métodos Numéricos & 10\\
Ingeniería de Software I & 7\\
Ingeniería de Software II & 9\\
Teoría de las Comunicaciones & 10\\
Bases de Datos & 9\\
Lógica y Computabilidad & 9\\
Teoría de Lenguajes & 10\\
Paradigmas de Lenguajes de Programación & 10\\
\end{tabular}

\begin{tabular}{l c}
\textbf{Optativas} \smallskip & \\
Redes Neuronales & 9\\
Introducción a Tecnologías del Habla & 9\\
Teoría de Juegos & Solo cursada\\
Desarrollo de Sistemas Operativos & 10\\
Aprendizaje Automático & 10\\
\end{tabular}

\bigskip

Escala de notas: 10

Promedio: 9.09

\end{document}
