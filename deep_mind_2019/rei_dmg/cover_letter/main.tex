%% start of file `template.tex'.
%% Copyright 2006-2013 Xavier Danaux (xdanaux@gmail.com).
%
% This work may be distributed and/or modified under the
% conditions of the LaTeX Project Public License version 1.3c,
% available at http://www.latex-project.org/lppl/.


\documentclass[11pt,a4paper,sans]{moderncv}        % possible options include font size ('10pt', '11pt' and '12pt'), paper size ('a4paper', 'letterpaper', 'a5paper', 'legalpaper', 'executivepaper' and 'landscape') and font family ('sans' and 'roman')

% moderncv themes
\moderncvstyle{classic}                             % style options are 'casual' (default), 'classic', 'oldstyle' and 'banking'
\moderncvcolor{blue}                               % color options 'blue' (default), 'orange', 'green', 'red', 'purple', 'grey' and 'black'
%\renewcommand{\familydefault}{\sfdefault}         % to set the default font; use '\sfdefault' for the default sans serif font, '\rmdefault' for the default roman one, or any tex font name
\nopagenumbers{}                                  % uncomment to suppress automatic page numbering for CVs longer than one page

% character encoding
\usepackage[utf8]{inputenc}                       % if you are not using xelatex ou lualatex, replace by the encoding you are using
%\usepackage{CJKutf8}                              % if you need to use CJK to typeset your resume in Chinese, Japanese or Korean

% adjust the page margins
\usepackage[scale=0.75]{geometry}
%\setlength{\hintscolumnwidth}{3cm}                % if you want to change the width of the column with the dates
%\setlength{\makecvtitlenamewidth}{10cm}           % for the 'classic' style, if you want to force the width allocated to your name and avoid line breaks. be careful though, the length is normally calculated to avoid any overlap with your personal info; use this at your own typographical risks...

% personal data
\name{Martin}{Miguel}
%\name{}{}
%\title{Resumé title}                               % optional, remove / comment the line if not wanted
\address{}{}{Buenos Aires, Argentina}% optional, remove / comment the line if not wanted; the "postcode city" and and "country" arguments can be omitted or provided empty
\phone[mobile]{+54~(11)~3181~6018}                   % optional, remove / comment the line if not wanted
%\phone[fixed]{+2~(345)~678~901}                    % optional, remove / comment the line if not wanted
%\phone[fax]{+3~(456)~789~012}                      % optional, remove / comment the line if not wanted
\email{m2.march@gmail.com}                               % optional, remove / comment the line if not wanted
%\homepage{www.johndoe.com}                         % optional, remove / comment the line if not wanted
%\extrainfo{additional information}                 % optional, remove / comment the line if not wanted
%\photo[64pt][0.4pt]{picture}                       % optional, remove / comment the line if not wanted; '64pt' is the height the picture must be resized to, 0.4pt is the thickness of the frame around it (put it to 0pt for no frame) and 'picture' is the name of the picture file
%\quote{Some quote}                                 % optional, remove / comment the line if not wanted

% to show numerical labels in the bibliography (default is to show no labels); only useful if you make citations in your resume
%\makeatletter
%\renewcommand*{\bibliographyitemlabel}{\@biblabel{\arabic{enumiv}}}
%\makeatother
%\renewcommand*{\bibliographyitemlabel}{[\arabic{enumiv}]}% CONSIDER REPLACING THE ABOVE BY THIS

% bibliography with mutiple entries
%\usepackage{multibib}
%\newcites{book,misc}{{Books},{Others}}
%----------------------------------------------------------------------------------
%            content
%----------------------------------------------------------------------------------
\begin{document}
%-----       letter       ---------------------------------------------------------
% recipient data
\recipient{DeepMind}{}
\date{}
\opening{Dear Sir or Madam,}
\closing{Yours faithfully,}
%\enclosure[Attached]{curriculum vit\ae{}}          % use an optional argument to use a string other than "Enclosure", or redefine \enclname
\makelettertitle

I am Martin Miguel and I am applying for the DeepMind Research Science
Internship for 2020. I am very interested in taking part on the current
developments in AI and Neuroscience. 

I am currently enrolled and three years into my PhD. I graduated with a
combined Bachelor + Master's degree in Computer Science at Universidad de
Buenos Aires in 2015 and continued into the Doctorate. My research topic is
developing models of cognition in time for exploring affect in music,
particularly rhythms. Theoretical frameworks posit that part of musical affect
comes from violation of expectations. Since music is highly structured and
regular, it is easy for it to generate strong expectations. My focus is on
having computational models of music cognition that integrate stimuli over time
and provide a certainty metric. Certainty on the expectations is key as only
violations of strong predictions generate affect.

During my PhD I have worked in modeling beat expectation, the most basic
rhythmic expectation. We took from previous work and produced an agent-based
model that had the previously mentioned features.  Later, we set to evaluate
whether the model's certainty measure related to cognition. We performed an
experiment where participants were asked to tap to the beat while listening to
rhythms of varying complexity. With the results, we could assert that our
model's certainty related to how strongly participants felt the beat and how
precisely they followed it. The details and evaluation of the model are
currently under review in a music research journal. The preliminary results of
the experiment were presented in the meeting of the Society for Music
Perception and Cognition in August 2019. 

The next step is expanding the modeling to capture hierarchical expectations in
music, maintaining the focus on continuous evaluation and certainty. The first
candidate are Hierarchical Bayesian Models. Their two-level inference
mechanic where a top-level grammar structures the inference from the
observable data allows describing the groups and repetitions seen in music.
Bayesian models have been used to continuously integrate information and
inherently allow calculating certainty. Another venue of exploration comes from
transformer deep neural networks. Transformers have successfully been used to
capture structure and long-time dependencies in sequential data, including text
and music. 
%Further research is required to see whether the inner workings of
%these networks can be used to predict listener's expectation certainty.

My approach to research has been question-centered and has turned me to
incorporate knowledge from areas outside computer science, such as experimental
psychology and neuroscience. I believe this makes my profile appealing to
DeepMind as I posses the technical background for AI and the hands-on
experience for research that involves thinking about cognition and the design
of experiments.

\vspace{0.4cm}
Thank you in advance for reviewing my application.
\vspace{0.2cm}

%understand how to design the experiment, what stimuli was valid and how to
%correctly frame our question and model in cognition and the theory of emotion.
%
%Humans are constantly reading their
%environment, trying to understand it and make predictions. Such predictions are
%made over-time, not only about what will happen but when it will happen.
%Moreover, our understanding makes use of abstractions, concepts used to
%summarize features we percieve into single objects. Finally, we also attribute
%a degree of certainty to our predictions, which are used to make future
%decisions. My interests are in advancing the computer modeling of human
%cognition of events in time. 
%
%I am currently developing my research at the
%Applied Artificial Intelligence Lab at Universidad de Buenos Aires. In our lab
%we apply machine learning techniques to real-life problems. Examples include
%speaker recognition, turn-taking in dialogs and prediction of mental states
%from text. The lab works interdisciplinary, having members from outside of the
%computer science field (e.g.: physics, biology and linguistics) and research in
%neuroscience topics such as predictabilty in text, patterns of information
%lookup from eye movement or changes in brain conectivity in kids after playing
%games.
%
%The origins of my interest come from affect in music: what are the mechanisms
%in music that makes it move us so deeply? Specifically, I was interested in the
%mechanisms that make rhythms entrancing, exhilarating, powerful. Reviewing
%previous work on the matter I reached theoretical frameworks that posit that
%part of musical affect comes from violation to expectations. Music is highly
%structure and regular, which makes it easy to create strong expectations about
%what will happen next. By playing with these expectations music manages to move
%us. With my background in Computer Science, I could contribute to the field by
%developing models of musical expectations that can be used to analyze in more
%depth such mechanisms of affect. In addition, the generality of the framework
%of musical expectation allows applying the develop models to new types of
%stimuli. Human productions for communication, in general, share the feature of
%structure, and as such the models can be applied to speech and text. Music is
%particularly structured and makes for an excellent testing ground for
%developing models of hierarchical abstractions, predictions in time and
%certainty.
%
%During my PhD I have worked in modeling beat expectation, the most basic
%rhythmic expectation. The focus was to make the model evolve in time and
%provide a measure of certainty, as they are needed for inspecting how
%expectation violation and affect is related. We took from previous work and
%adapted an agent-based model for the task. Later, we set to evaluate whether
%the model's certainty measure related to cognition. We performed an experiment
%where participants were asked to tap to the beat to rhythms of varying
%complexity. With the results, we could assert that our model's certainty
%related to how strongly participants felt and followed the beat of the stimuli.
%The details and evaluation of the model are currently under review in a music
%research journal. The preliminary results of the experiment were presented in
%the meeting of the Society for Music Perception and Cognition in August 2019.
%The development of this first phase of our work required incorporating
%knowledge from areas outside computer science. It was necessary to review work
%on experimental psychology to understand how to design the experiment, what
%stimuli was valid and how to correctly frame our question and model in
%cognition and the theory of emotion.
%
%My main next step is expanding the modeling to capture hierarchical
%expectations in music, mantaining the focus on continous evaluation and
%certainty. The first natural candidate are Hierarchical Bayesian Models. Their 
%two-level inference mechanic where a top-level grammar structures the inference
%towards the observable data allows describing the groups and repetitions seen
%in music. Bayesian models have been used to continously integrate information
%and inherently allow calculating certainty. Another venue of exploration comes
%from transformer deep neural networks. Transformers have successfully been used
%to capture structure and long-time dependencies in sequenced data, including
%text and music. 


\makeletterclosing

\end{document}


%% end of file `template.tex'.
