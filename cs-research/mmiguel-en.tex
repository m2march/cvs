%%%%%%%%%%%%%%%%%%%%%%%%%%%%%%%%%%%%%%%
% Deedy CV/Resume
% XeLaTeX Template
% Version 1.0 (5/5/2014)
%
% This template has been downloaded from:
% http://www.LaTeXTemplates.com
%
% Original author:
% Debarghya Das (http://www.debarghyadas.com)
% With extensive modifications by:
% Vel (vel@latextemplates.com)
%
% License:
% CC BY-NC-SA 3.0 (http://creativecommons.org/licenses/by-nc-sa/3.0/)
%
% Important notes:
% This template needs to be compiled with XeLaTeX.
%
%%%%%%%%%%%%%%%%%%%%%%%%%%%%%%%%%%%%%%

\documentclass[a4paper]{deedy-resume} % Use US Letter paper, change to a4paper for A4 

\newcommand{\jobtitle}[1]{\runsubsection{\normalsize{#1}}}

\newcommand{\jobitem}[4]{%
    %(title, place, date range, description)
\vbox{
\jobtitle{#1}\descript{| #2}
\location{#3}
#4
}
\sectionspace
}

\newcommand{\publication}[3]{%(title, authors, description)
\vbox{
    \descript{#1}
    \location{#2}
    #3
}
\sectionspace
}

\begin{document}

%----------------------------------------------------------------------------------------
%	TITLE SECTION
%----------------------------------------------------------------------------------------

%\lastupdated % Print the Last Updated text at the top right

\namesection{Martin}{Miguel}{ % Your name
    \urlstyle{same}\href{https://mmiguel.liaa.dc.uba.ar}{mmiguel.liaa.dc.uba.ar} \\ % Your website, LinkedIn profile or other web address
\href{mailto:mmiguel@dc.uba.ar}{mmiguel@dc.uba.ar} \\
+541131816018 \\
Buenos Aires, Argentina
}

%------------------------------------------------
% Statement 
%------------------------------------------------

%\section{Mission Statement}

%\begin{flushleft}
%I have always been driven by curiosity and problem-solving. This drive took me
%to study Computer Science as a mean to gain the ability to ask good
%questions, find rigorous answers and carry them into real world solutions.
%Throughout my career I have developed both my applied and academic skills. I
%have worked in industry and acquired methodology and know-how and I am now
%working on my PhD degree, learning from research papers and performing
%experiments. Having curiosity as my main motor I have outed from strict
%computer science into empirical sciences, mainly neurosciences. I am
%interested in human perception and the interaction with each other and the
%world. I am currently trying to add knowledge about how we process the world
%and create expectations about the future. More specifically, I am working on
%modeling musical expectations, as it is a familiar
%semi-structured stimuli that makes a good first step in that direction.
%\end{flushleft}

%------------------------------------------------
% Education
%------------------------------------------------

\section{Education} 

\descript{PhD in Computer Science}
\location{April 2016 - (Expected) May 2022 | University of Buenos Aires (UBA), Buenos Aires, Argentina}
Advisor: Diego Fernandez Slezak - Applied Artificial Intelligence Lab (LIAA),
Computer Science Department, University of Buenos Aires, Buenos Aires, Argentina;
Computer Science Institute, National Scientific and Technical Research Council
(CONICET)-UBA, Argentina \\
Co-advisor: Mariano Sigman - Neuroscience Laboratory, Torcuato Di
Tella University, Buenos Aires, Argentina; Faculty of 
Language and Education, Nebrija University, Madrid, Spain}

\sectionspace

\descript{Professional Musician}
\location{April 2015 - (Paused) June 2017 | Contemporary Music School, Buenos Aires,
Argentina}

\sectionspace

\descript{BS + MS in Computer Science}
\location{April 2008 - December 2015 | University of Buenos Aires, Buenos Aires, Argentina}

\sectionspace



%------------------------------------------------
% Research
%------------------------------------------------

\section{Research}

\subsection{Publications}
\publication{(In preparation) Grammar-based modeling of rhythmic perception}
{Martin A. Miguel, Mariano Sigman, Diego Fernandez Slezak}
{Paper describing a model of beat and meter expectation using grammar-based
  bayesian inference. 
%To be presented in \emph{Music Cognition Journal}.
}

\publication{(In press) Modeling beat uncertainty as a 2D distribution of period and
phase: a MIR task proposal}
{Martin A. Miguel,  Diego Fernandez Slezak}
{Paper describing a methodology to model beat uncertainty considering period
  and phase from free tapping data and an evaluation criterion for MIR models. 
Proc. of the 22nd Int. Society for Music Information
Retrieval Conf., Online, 2021.
%To be presented in \emph{Music Cognition Journal}.
}

\publication{(In press) Pulse clarity metrics developed from a deep learning beat tracking
model}
{Nicolas Pironio, Diego Fernandez Slezak, Martin A. Miguel}
{Paper describing metrics of pulse clarity obtained from modifications to a
neural-network based beat tracking model. 
Proc. of the 22nd Int. Society for Music Information
Retrieval Conf., Online, 2021.
%Under review in \emph{Music Information Retrieval Journal}.
}

\publication{A simple and cheap setup for timing tapping responses synchronized
to auditory stimuli}
{Martin A. Miguel, Pablo Riera, Diego Fernandez Slezak}
{Paper describing an experimental setup for capturing timing of
tapping responses synchronized against auditory stimuli. The setup requires
minimal programming skills and uses unexpensive equipment. Behav Res (2021).
https://doi.org/10.3758/s13428-021-01653-y
}

\publication{From beat tracking to beat expectation: Cognitive-based beat
tracking for capturing pulse clarity through time}
{Martin A. Miguel, Mariano Sigman, Diego Fernandez Slezak}
{Paper presenting a model of beat tracking adapted to produce a metric of
pulse-clarity over time. (2020) PLoS ONE 15(11): e0242207. 
https://doi.org/10.1371/journal.pone.0242207} 

\publication{Mate Marote: a BigData platform for massive scale
educational interventions.}
{Laouen Belloli, Martín A. Miguel, Andrea P. Goldin and Diego Fernández Slezak}
{Paper describing a web platform that hosts and collects data from educational
games. 45-JAIIO, 2016, Buenos Aires, Argentina (ISSN: 2451-7569, p107-114).}

%\publication{Inferencia de Tactus con Fundamentos Estadı́sticos para Tap-dancing.}
%{Martin A. Miguel}
%{44-JAIIO, 2015, Rosario, Argentina (ISSN: 2451-7585).}
%
\subsection{Conferences and Schools \hfill}

\publication{Modeling beat ambiguity in period and phase}
{Martin A. Miguel, Diego Fernandez Slezak}
{Poster presenting a methodology from gathering a beat distribution from free
tapping data.
Presented in the International Conference of Students of Systematic
Musicology 21, Online and Aahrus, Denmark, 2021}

\publication{A continuous model of pulse clarity: towards inspecting affect
through expectations in time (Updated)}
{Martin A. Miguel, Mariano Sigman, Diego Fernandez Slezak}
{Poster describing an updated evaluation of our beat expectation model's 
  measure of pulse clarity considering new data and constrating models.
Presented in Neuromusic VII, Online and Aahrus, Denmark, 2021. }

\publication{Evaluation of pulse clarity models on multiple datasets}%
{Nicolas Pironio, Diego Fernandez Slezak, Martin A. Miguel}
{Poster presenting the evaluation of multiple pulse clarity models.
Presented at the Rhythm Perception and Production Workshop 2021, Online and
Oslo, Norway, 2021}

\publication{Development and evaluation of pulse clarity metrics based of a
deep learning beat tracking model}%
{Nicolas Pironio, Diego Fernandez Slezak, Martin A. Miguel}
{Poster presenting metrics of pulse clarity obtained from modifications to a
neural-network based beat tracking model.
16th International Conference on Music Perception and Cognition, Online 2021}

\publication{Tapping to your own beat: experimental setup for exploring
subjective tacti distribution and pulse clarity}
{Martin A. Miguel, Mariano Sigman, Diego Fernandez Slezak}
{Poster describing a novel experimental setup that extends on previous methods
allowing exploration of subjective tacti on top of pulse clarity.
Presented in SMPC 2019, New York, USA (DOI 10.17605/OSF.IO/7SQAW).}

\publication{Tapping to your own beat: experimental setup for exploring
subjective tacti distribution and pulse clarity}
{Martin A. Miguel, Mariano Sigman, Diego Fernandez Slezak}
{Poster describing a novel experimental setup that extends on previous methods
allowing exploration of subjective tacti on top of pulse clarity.
Presented in SMPC 2019, New York, USA (DOI 10.17605/OSF.IO/7SQAW).}

\publication{A continuous model of pulse clarity: towards inspecting affect
through expectations in time}
{Martin A. Miguel, Mariano Sigman, Diego Fernandez Slezak}
{Poster describing how our beat expectation model's measure of pulse
clarity relates with pulse clarity extracted from empirical data.
Presented in SMPC 2019, New York, USA (DOI 10.17605/OSF.IO/FGVB2).}

\publication{Assistance to KHIPU 2019}{University of the Republic, 
Montevideo, Uruguay}

\publication{Assistance to the meeting of the Society of Music Perception and
Cognition (SMPC 2019)}{New York University, New York, USA}{}

\publication{Assistance to Machine Learning Summer School (MLSS 2018)}{Torcuato
Di Tella University, Buenos Aires, Argentina}{}

\publication{Assistance and volunteering at IJCAI 2015}{Buenos Aires,
Argentina}{}

\subsection{Mentorships \hfill}
\publication{Mentor of undergraduate research internship: Exploration of music
style transfer techniques based on VAEs latent spaces from symbolic music data}
{April 2021 - April 2022}

\publication{Mentor of undergraduate research internship: Analysis of the
behaviour of a beat tracking model to estimate pulse clarity}
{April 2020 - April 2021}


\subsection{Scholarships \hfill}
\publication{PhD Grant | National Scientific and Technical Research Council
(CONICET), Argentina}
{April 2016 - April 2021}


%------------------------------------------------
% Experience
%------------------------------------------------

\section{Teaching Experience}

\jobitem{Teaching Fellow}{Universidad de Buenos Aires}
{April 2016 - Currently}
%{Teaching fellow of the \emph{Algorithms and Data Structures II} course.}

\jobitem{Teaching Assistant}{Universidad de Buenos Aires}
{March 2011 - July 2012}
%{Teaching assistant of the \emph{Algorithms and Data Structures II} course.}

\section{Industry Experience}

\jobitem{Data Scientist}{Avenida.com}
{January 2016 - March 2016}
%{Improvement of search engine configuration, implementation of
%\emph{search-as-you-type} features and assistance in team management.}

\jobitem{Software Engineer (Online Educational Games)}{MateMarote Project}
{April 2015 - December 2015}
%{Development of java backend infrastructure and javascript videogames for a
%neuroscientifically based educational software.}

\jobitem{Software Engineer Intern}{Google.com}
{January 2014 - April 2014}
%{Development and extensions of testing frameworks for
%performance, end-to-end and regression tests.}

\jobitem{Java Programmer}{Despegar.com}
{August 2012 - December 2013}
%{Development of components integrating a larger application system. Development
%of web applications and utility frameworks.}

%\jobitem{Assistant Professor}{Universidad de Buenos Aires}
%{July 2012 - March 2011}
%{Assistant Professor of \emph{Algorithms and Data Structures I \& II}.}

\jobitem{Jr. Java Programmer (J2ME / Blackberry)}{SenseByte}
{January 2009 - January 2010}
%{Development of both stand-alone and client-server applications. Development of
%applications interfacing with non-standard hardware.}

\sectionspace

%------------------------------------------------
% Coursework
%------------------------------------------------

\section{Coursework}

\small
\small
\subsection{Graduate}
\begin{minipage}[t]{0.48\textwidth}
    \phantom{Hola}

\hspace{1cm}\begin{tabular}{p{0.6\textwidth} p{0.25\textwidth}}
Calculus & 9 \\
Algebra & 5 \\
Probability and Statistics & 10 \\
Algorithms and Data Structures I & 10 \\
Algorithms and Data Structures II & 10 \\
Algorithms and Data Structures III & 9 \\
Computer System Architecture I & 8 \\
Computer System Architecture II & 8 \\
Operating Systems & 10 \\
Numerical Methods & 10 \\
Software Engineering I & 7 \\
Software Engineering II & 9 \\
\end{tabular}
\end{minipage}
\begin{minipage}[t]{0.48\textwidth}
    \phantom{Hola}

\begin{tabular}{p{0.6\textwidth} p{0.25\textwidth}}
Systems Networks & 10 \\
Database Systems & 9 \\
Logic and Computability Theory & 9 \\
Language Theory & 10 \\
Programming Paradigms & 10 \\
Neural Networks & 9 \\
Introduction to Speech Technologies & 9 \\
Game Theory & Assisted Only \\
Operating Systems Development & 10 \\
Machine Learning & 10 \\
Master's Thesis & 10 \\
\end{tabular}
\end{minipage}

\sectionspace

\hspace{1cm}\begin{tabular}{l p{0.25\textwidth}}
Graduate GPA & 9.14 \\
Grade Scale & 10 \\
\end{tabular}

\sectionspace
\subsection{Doctorate}
\begin{minipage}{\textwidth}
\hspace{1cm}\begin{tabular}{p{0.7\textwidth} p{0.25\textwidth}}
Introduction to Data Science &  \\
Data Science in R &  \\
Bayesian Inference &  \\
Integration of Knowledge Bases &  \\
Signal Processing &  \\
Introduction to Computational Cognitive Neuroscience &  \\
\end{tabular}
\end{minipage}


\sectionspace % Some whitespace after the section



\end{document}
