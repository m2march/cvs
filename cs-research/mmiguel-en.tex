%%%%%%%%%%%%%%%%%%%%%%%%%%%%%%%%%%%%%%%
% Deedy CV/Resume
% XeLaTeX Template
% Version 1.0 (5/5/2014)
%
% This template has been downloaded from:
% http://www.LaTeXTemplates.com
%
% Original author:
% Debarghya Das (http://www.debarghyadas.com)
% With extensive modifications by:
% Vel (vel@latextemplates.com)
%
% License:
% CC BY-NC-SA 3.0 (http://creativecommons.org/licenses/by-nc-sa/3.0/)
%
% Important notes:
% This template needs to be compiled with XeLaTeX.
%
%%%%%%%%%%%%%%%%%%%%%%%%%%%%%%%%%%%%%%

\documentclass[a4paper]{deedy-resume} % Use US Letter paper, change to a4paper for A4 

\newcommand{\jobtitle}[1]{\runsubsection{\normalsize{#1}}}

\newcommand{\jobitem}[4]{%
    %(title, place, date range, description)
\vbox{
\jobtitle{#1}\descript{| #2}
\location{#3}
#4
}
\sectionspace
}

\newcommand{\publication}[3]{%(title, authors, description)
\vbox{
    \descript{#1}
    \location{#2}
    #3
}
\sectionspace
}

\begin{document}

%----------------------------------------------------------------------------------------
%	TITLE SECTION
%----------------------------------------------------------------------------------------

%\lastupdated % Print the Last Updated text at the top right

\namesection{Martin}{Miguel}{ % Your name
\urlstyle{same}\url{http://liaa.dc.uba.ar} \\ % Your website, LinkedIn profile or other web address
\href{mailto:mmiguel@dc.uba.ar}{mmiguel@dc.uba.ar} \\
+541131816018 \\
CABA, Buenos Aires, Argentina
}

%------------------------------------------------
% Statement 
%------------------------------------------------

\section{Mission Statement}

\begin{flushleft}
I have always been driven by curiosity and problem-solving. This drive took me
to study Computer Science as a mean to gain the ability to ask good
questions, find rigorous answers and carry them into real world solutions.
Throughout my career I have developed both my applied and academic skills. I
have worked in industry and acquired methodology and know-how and I am now
working on my PhD degree, learning from research papers and performing
experiments. Having curiosity as my main motor I have outed from strict
computer science into empirical sciences, mainly neurosciences. I am
interested in human perception and the interaction with each other and the
world. I am currently trying to add knowledge about how we process the world
and create expectations about the future. More specifically, I am working on
modeling musical expectations, as it is a familiar
semi-structured stimuli that makes a good first step in that direction.
\end{flushleft}

%------------------------------------------------
% Education
%------------------------------------------------

\section{Education} 

\descript{PhD in Computer Science}
\location{Expected April 2021 - April 2016 | UBA, Buenos Aires, Argentina}
Advisor: Diego Fernandez Slezak - LIAA, DC, UBA, Buenos Aires, Argentina; ICC, CONICET-UBA, Argentina \\
Co-advisor: Mariano Sigman - LNI, UTDT, Buenos Aires, Argentina

\sectionspace

\descript{Professional Musician}
\location{Paused (June 2017) - April 2015 | Escuela de Música Contemporánea, Buenos Aires,
Argentina}

\sectionspace

\descript{BS + MS in Computer Science}
\location{December 2015 - April 2008 | UBA, Buenos Aires, Argentina}

\sectionspace


%------------------------------------------------
% Experience
%------------------------------------------------

\section{Experience}

\jobitem{Teaching Fellow}{Universidad de Buenos Aires}
{Currently - April 2016 | July 2012 - March 2011}
{Assistant Professor of \emph{Algorithms and Data Structures II}.}


\jobitem{Data Scientist}{Avenida.com}
{March 2016 - January 2016}
{Improvement of search engine configuration, implementation of
\emph{search-as-you-type} features and assistance in team management.}

\jobitem{Software Engineer}{MateMarote}
{December 2015 - April 2015}
{Development of java backend infrastructure and javascript videogames for a
neuroscientifically based educational software.}

\jobitem{Software Engineer Intern}{Google.com}
{April 2014 - January 2014}
{Development and extensions of testing frameworks for
performance, end-to-end and regression tests.}

\jobitem{Java Programmer}{Despegar.com}
{December 2013 - August 2012}
{Development of components integrating a larger application system. Development
of web applications and utility frameworks.}

%\jobitem{Assistant Professor}{Universidad de Buenos Aires}
%{July 2012 - March 2011}
%{Assistant Professor of \emph{Algorithms and Data Structures I \& II}.}

\jobitem{Jr. Java Programmer (J2ME / Blackberry)}{SenseByte}
{January 2010 - January 2009}
{Development of both stand-alone and client-server applications. Development of
applications interfacing with non-standard hardware.}


%------------------------------------------------
% Research
%------------------------------------------------

\section{Research}

\subsection{Publications}

\publication{(In preparation) Simple framework for rhythmic asynchrony experiments in Python}
{Martin A. Miguel, Diego Fernandez Slezak, Mariano Sigman}
{Paper describing an experimental setup for tapping used to measure
synchronization of responses against the stimuli. The setup requires minimal 
not-specialized equipment. To be published in \emph{Behavioural Research
Methods}.}

\publication{(In preparation) Bayesian beat expectation and uncertainty}
{Martin A. Miguel, Diego Fernandez Slezak, Mariano Sigman}
{Paper describing evaluation of beat expectation model against experimentally 
captured responses of tapping. To be published in \emph{Cognitive Science
Journal}.}

\publication{From beat tracking to beat expectation}
{Martin A. Miguel, Diego Fernandez Slezak, Mariano Sigman}
{(Under review) Journal of New Music Research}

\publication{Tapping to your own beat: experimental setup for exploring
subjective tacti distribution and pulse clarity}
{Martin A. Miguel, Mariano Sigman, Diego Fernandez Slezak}
{Poster describing a novel experimental setup that extends on previous methods
allowing exploration of subjective tacti on top of rhythmic complexity.
Presented in SMPC 2019, New York, USA (DOI 10.17605/OSF.IO/7SQAW).}

\publication{A continuous model of pulse clarity: towards inspecting affect
through expectations in time}
{Martin A. Miguel, Mariano Sigman, Diego Fernandez Slezak}
{Poster describing how our beat expectation model's measure of rhythmic
complexity related with complexity extracted from empirical data.  Presented in
SMPC 2019, New York, USA (DOI 10.17605/OSF.IO/FGVB2).}

\publication{Mate Marote: a BigData platform for massive scale
educational interventions.}
{Martin A. Miguel}
{45-JAIIO, 2016, Buenos Aires, Argentina (ISSN: 2451-7569).}

%\publication{Inferencia de Tactus con Fundamentos Estadı́sticos para Tap-dancing.}
%{Martin A. Miguel}
%{44-JAIIO, 2015, Rosario, Argentina (ISSN: 2451-7585).}
%
\subsection{Conferences and Schools \hfill}
\publication{Assistance to KHIPU 2019}{Universidad de la República, 
Montevideo, Uruguay}

\publication{Assistance to the meeting of the Society of Music Perception and
Cognition (SMPC 2019)}{Universidad Torcuato Di Tella, Buenos
Aires, Argentina}{}

\publication{Assistance to Machine Learning Summer School (MLSS 2018)}{Universidad Torcuato Di Tella, Buenos
Aires, Argentina}{}

\publication{Assistance and volunteering at IJCAI 2015}{Buenos Aires,
Argentina}{}

\section{Scholarships}

\descript{PhD Grant | CONICET, Argentina}
\location{April 2021 - April 2016}

\sectionspace

%------------------------------------------------
% Coursework
%------------------------------------------------

\section{Coursework}

\small
\small
\subsection{Graduate}
\begin{minipage}[t]{0.48\textwidth}
    \phantom{Hola}

\hspace{1cm}\begin{tabular}{p{0.6\textwidth} p{0.25\textwidth}}
Calculus & 9 \\
Algebra & 5 \\
Probability and Statistics & 10 \\
Algorithms and Data Structures I & 10 \\
Algorithms and Data Structures II & 10 \\
Algorithms and Data Structures III & 9 \\
Computer System Architecture I & 8 \\
Computer System Architecture II & 8 \\
Operating Systems & 10 \\
Numerical Methods & 10 \\
Software Engineering I & 7 \\
Software Engineering II & 9 \\
\end{tabular}
\end{minipage}
\begin{minipage}[t]{0.48\textwidth}
    \phantom{Hola}

\begin{tabular}{p{0.6\textwidth} p{0.25\textwidth}}
Systems Networks & 10 \\
Database Systems & 9 \\
Logic and Computability Theory & 9 \\
Language Theory & 10 \\
Programming Paradigms & 10 \\
Neural Networks & 9 \\
Introduction to Speech Technologies & 9 \\
Game Theory & Assisted Only \\
Operating Systems Development & 10 \\
Machine Learning & 10 \\
Master's Thesis & 10 \\
\end{tabular}
\end{minipage}

\sectionspace

\hspace{1cm}\begin{tabular}{l p{0.25\textwidth}}
Graduate GPA & 9.14 \\
Grade Scale & 10 \\
\end{tabular}

\sectionspace
\subsection{Doctorate}
\begin{minipage}{\textwidth}
\hspace{1cm}\begin{tabular}{p{0.7\textwidth} p{0.25\textwidth}}
Introduction to Data Science &  \\
Data Science in R &  \\
Bayesian Inference &  \\
Integration of Knowledge Bases &  \\
Signal Processing &  \\
Introduction to Computational Cognitive Neuroscience &  \\
\end{tabular}
\end{minipage}


\sectionspace % Some whitespace after the section



\end{document}
