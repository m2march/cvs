%-------------------------------------------------------------------------------
%	SECTION TITLE
%-------------------------------------------------------------------------------
\cvsection{Main Publications}


%-------------------------------------------------------------------------------
%	CONTENT
%-------------------------------------------------------------------------------

%---------------------------------------------------------
\cvsubsection{Published}
%---------------------------------------------------------

\begin{cvpubs}
  \cvpub{\textbf{Miguel, M.A.} and Fernandez Slezak, D. (2021). Modeling beat uncertainty as a 2D distribution of period and
phase: a MIR task proposal. Proc. of the 22nd Int. Society for Music Information
Retrieval Conf., Online.}{Paper describing a methodology to model beat uncertainty considering period
and phase from free tapping data and an evaluation criterion for MIR models.}

  \cvpub{Pironio, N., Fernandez Slezak, D. and \textbf{Miguel, M.A.} (2021) Pulse clarity metrics developed from a deep learning beat tracking
model. 
Proc. of the 22nd Int. Society for Music Information
Retrieval Conf., Online, 2021.}%
{Paper describing metrics of pulse clarity obtained from modifications to a
neural-network based beat tracking model. }

\cvpub{\textbf{Miguel, M.A.}, Riera, P., and Fernandez Slezak, D. (2021) A simple and cheap setup for timing tapping responses synchronized
to auditory stimuli. Behav Res. https://doi.org/10.3758/s13428-021-01653-y
}{Paper describing an experimental setup for capturing timing of
tapping responses synchronized against auditory stimuli. The setup requires
minimal programming skills and uses unexpensive equipment.}

\cvpub{\textbf{Miguel, M.A.}, Sigman, M. and Fernandez Slezak, D. (2020) From beat tracking to beat expectation: Cognitive-based beat
tracking for capturing pulse clarity through time. PLoS ONE 15(11): e0242207.
https://doi.org/10.1371/journal.pone.0242207}%
{Paper presenting a model of beat tracking adapted to produce a metric of
pulse-clarity over time.} 

\end{cvpubs}


% %---------------------------------------------------------
%\cvsubsection{In Review}
% %---------------------------------------------------------

%\begin{cvpubs}
%  \cvpub{Manuscript 1}{}
%
%  \cvpub{Manuscript 2}{}
%\end{cvpubs}

% %---------------------------------------------------------
\cvsubsection{In Prep}
% %---------------------------------------------------------

\begin{cvpubs}
\small \color{black}
\cvpub{Grammar-based modeling of rhythmic perception}
{Paper describing a model of beat and meter expectation using grammar-based
  bayesian inference.
To be presented in \emph{Music Cognition Journal}.
}

\end{cvpubs}

% %---------------------------------------------------------
