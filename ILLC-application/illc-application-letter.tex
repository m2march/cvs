\documentclass[a4paper]{letter}

\usepackage{hyperref}
\usepackage{todonotes}

\setlength\topmargin{-70pt}
\setlength\textheight{\textheight+120pt}

\signature{Martin A. Miguel}

\begin{document}

\begin{letter}{}

\opening{Dear selection committee:}

Hello, my name is Martin Miguel. I am a soon-to-be Computer Science Master from
the Universidad of Buenos Aires, Argentina. I am currently wrapping up my
Master's Thesis on music cognition topics. My thesis topic is modeling tactus
inference (or BI) but looking for a model that can provide insight on how we
synchronize with music when we hear it. I heard of the doctoral opportunity by
following the music cognition blog by Prof. Dr. H. Honing.

I am currently being advised by Dr. Diego Fernandez Slezak from the Applied
Artificial Intelligence Lab at UBA.%
\footnote{\url{http://liaa.dc.uba.ar/}}
He can be reached for references at
\href{mailto:dfslezak@dc.uba.ar}{\url{dfslezak@dc.uba.ar}}.

I got into the field of music cognition out of curiosity for the musical
processes that happen in a listener's mind. I have been dancing and playing
music for over 10 years now and I am very curious about what's the magic that
makes us twitch. In particular, I have been learning to tap dance for the last
4 years and I have found some rhythmical passages that are particularly
evocative. This is not common in tap dancing passages and I am afraid that
comes from lack of theoretical knowledge on how to produce rhythms. The issue
lead to the idea of trying to model some of the rhythmical cognitive processes
in play in order to develop a tool that can help tap musicians get closer to
their own productions. In particular, I want to make situations of tension and
unexpectedness in music explicit, being this what I believe makes this passages
so interesting. The focus is on rhythmical expectations, i.e. expectations on
the timing of musical events.

% After a lot of rambling I ended up with a model for tactus inference. This is
% an overly studied subject but I decided to face it again since most models I
% had encountered (Temperley, Goto, Rosenthal, Schloss) seemed to be based on
% musical rules that were known to apply in western music, but it was not clear
% if they worked in other musical styles. In particular, tap dancing takes a lot
% from african percussive music, which does work differently. With that in mind I
% stumbled upon Huron (Sweet Anticipation, 2006), whose theoretical view is more
% general. He speaks of the music cognition process as a task of prediction and
% understanding. This aligns with my hypothesis of how we create expectation and
% how tension related issues arise in music. 

% The model I developed has two main characteristics. Firstly, it models the
% inner tactus expectation as a precise clock. This comes from the fact that
% musicians train with metronomes, which are precise clocks, and are sensible to
% very small asynchronisms. I believe part of the musical experience, especially
% in rich rhythmical passages, comes from this sensibility. I need to observe the
% contrast between an expressive event and a precise one to elaborate on the
% musical experience. Secondly, my model was built to produce output continuously
% as the passage is analysed. This is important since the expectation process
% when listening to music happens while listening to the song. The twitches and
% surprises happen in the middle of it. In particular, the model constantly
% produces tactus hypotheses and contrasts them by how well they predict the
% passage. I believe one possible source of tension in music comes from the lack
% of a strong feeling of structure - in this case the constant guiding pulse. The
% model I developed can be thought out as a re-imagining of Povel and Essens work
% adapted to obtain continuous feedback and manage imprecise events.

% The model has already been tested as a tactus inference system and works well.
% I have contrasted it with the Melisma system [Temperley, 2001] and it
% outperformed it in tap dancing transcriptions. I am still working on getting
% the data to draw some conclusion on whether the continuous feedback provided by
% the model is useful to predict features in our process of understanding the
% rhythm, such as uncertainty or expectation mismatch.

I will be graduating with a Bs + Ms by the end of June. I am very interested 
in further pursuing my line of research, but I'm looking towards doing so among
people who know the subject better. Sadly, there is no one else at UBA working
on these topics. As a future view, I'd like to enhance my model (or work on
another one) to try to capture the whole human process of understanding
rhythmicality. I'd also like to apply this rhythm perception model to other
situations such as the timing of speech in drama performances, oral
storytelling and stand-up comedy.

The research goals within ILLC match my interests perfectly. The
music cognition group\footnote{\url{http://cf.hum.uva.nl/mmm/}} shares my
interest in expressive performances and their cognitive interpretation.%
\footnote{Honing, H. (2013).  The structure and interpretation of rhythm in
music. In Deutsch, D. (ed.), Psychology of Music, 3rd edition (pp.
369-404). London: Academic Press.} It also shares part of my methodology:
computational modeling based on statistical tools.%
\footnote{Sadakata. M., Desain, P. \& Honing, H. (2006). The Bayesian way
to relate rhythm perception and production. Music Perception, 23(3), 267-286.
(c) 2006} As further reference, a work in progress paper by myself can be
read \href{
https://drive.google.com/file/d/0BzNsqva23xUGal9Bdl8xYWJOeHM/view?usp=sharing}%
{here}.

I am hoping that through this application a conversation can be started towards
how related my research interests are with those of the MCG and how I can be of
assistance to it as well as how it can help me grow and pursue my goals. 

\closing{Yours sincerely,} 

\end{letter}

\end{document}
